%%%%%%%%%%%%%%%%%%%%%%%%%%%%%%%%%%%%%%%
% Deedy - One Page Two Column Resume
% LaTeX Template
% Version 1.1 (30/4/2014)
%
% Original author:
% Debarghya Das (http://debarghyadas.com)
%
% Original repository:
% https://github.com/deedydas/Deedy-Resume
%
% IMPORTANT: THIS TEMPLATE NEEDS TO BE COMPILED WITH XeLaTeX
%
% This template uses several fonts not included with Windows/Linux by
% default. If you get compilation errors saying a font is missing, find the line
% on which the font is used and either change it to a font included with your
% operating system or comment the line out to use the default font.
% 
%%%%%%%%%%%%%%%%%%%%%%%%%%%%%%%%%%%%%%
% 
% TODO:
% 1. Integrate biber/bibtex for article citation under publications.
% 2. Figure out a smoother way for the document to flow onto the next page.
% 3. Add styling information for a "Projects/Hacks" section.
% 4. Add location/address information
% 5. Merge OpenFont and MacFonts as a single sty with options.
% 
%%%%%%%%%%%%%%%%%%%%%%%%%%%%%%%%%%%%%%
%
% CHANGELOG:
% v1.1:
% 1. Fixed several compilation bugs with \renewcommand
% 2. Got Open-source fonts (Windows/Linux support)
% 3. Added Last Updated
% 4. Move Title styling into .sty
% 5. Commented .sty file.
%
%%%%%%%%%%%%%%%%%%%%%%%%%%%%%%%%%%%%%%%
%
% Known Issues:
% 1. Overflows onto second page if any column's contents are more than the
% vertical limit
% 2. Hacky space on the first bullet point on the second column.
%
%%%%%%%%%%%%%%%%%%%%%%%%%%%%%%%%%%%%%%

\documentclass[]{deedy-resume-openfont}


\begin{document}

%%%%%%%%%%%%%%%%%%%%%%%%%%%%%%%%%%%%%%
%
%     09-03-2020
%
%%%%%%%%%%%%%%%%%%%%%%%%%%%%%%%%%%%%%%


%%%%%%%%%%%%%%%%%%%%%%%%%%%%%%%%%%%%%%
%
%     J. CATRIEL LOPEZ - CV
%
%%%%%%%%%%%%%%%%%%%%%%%%%%%%%%%%%%%%%%


\namesection{J. Catriel}{Lopez}{ 
\href{mailto:jcatriel.lopez@gmail.com}{jcatriel.lopez@gmail.com} | +54 0249 15 4528397 | Tandil, Buenos Aires, Argentina
}

%%%%%%%%%%%%%%%%%%%%%%%%%%%%%%%%%%%%%%
%
%     COLUMN ONE
%
%%%%%%%%%%%%%%%%%%%%%%%%%%%%%%%%%%%%%%

\begin{minipage}[t]{0.33\textwidth} 

%%%%%%%%%%%%%%%%%%%%%%%%%%%%%%%%%%%%%%
%     LINKS
%%%%%%%%%%%%%%%%%%%%%%%%%%%%%%%%%%%%%%

\section{Links} 
Github: \href{https://github.com/JCatrielLopez}{\custombold{@JCatrielLopez}} \\
LinkedIn: \href{https://www.linkedin.com/in/JCatrielLopez/}{\custombold{@JCatrielLopez}} \\
Dev.to: \href{https://dev.to/catriel}{\custombold{@Catriel}} \\
\sectionsep

%%%%%%%%%%%%%%%%%%%%%%%%%%%%%%%%%%%%%%
%     EDUCATION
%%%%%%%%%%%%%%%%%%%%%%%%%%%%%%%%%%%%%%

\section{Education} 

\subsection{Systems Engineer}
\descript{U. N. I. C. E. N}
\location{Expected 2021}

%
%\sectionsep
%\subsection{High School}
%\descript{Sagrada Familia}
%\location{2013}
%
\sectionsep


%%%%%%%%%%%%%%%%%%%%%%%%%%%%%%%%%%%%%%
%     SKILLS
%%%%%%%%%%%%%%%%%%%%%%%%%%%%%%%%%%%%%%

\section{Skills}
\subsection{Programming and Tools}
\location{Programming Languages:}
Python \textbullet{}   Java \textbullet{}  R \\
\location{Tools}
Pandas \textbullet{} ggplot2 \textbullet{} NumPy \textbullet{} Excel \textbullet{} 
Shiny \textbullet{} PostgreSQL \textbullet{} Tensorflow \textbullet{} Keras \textbullet{} Scrapy \textbullet{} bs4 \\
\location{Others}
Bash \textbullet{} Linux
\sectionsep

%%%%%%%%%%%%%%%%%%%%%%%%%%%%%%%%%%%%%%
%     LANGUAGES
%%%%%%%%%%%%%%%%%%%%%%%%%%%%%%%%%%%%%%

\section{Languages}
\location{Native:} Spanish \\
\location{Proficient:} English
\sectionsep

%%%%%%%%%%%%%%%%%%%%%%%%%%%%%%%%%%%%%%
%
%     COLUMN TWO
%
%%%%%%%%%%%%%%%%%%%%%%%%%%%%%%%%%%%%%%

\end{minipage} 
\hfill
\begin{minipage}[t]{0.66\textwidth} 

%%%%%%%%%%%%%%%%%%%%%%%%%%%%%%%%%%%%%%
%     EXPERIENCE
%%%%%%%%%%%%%%%%%%%%%%%%%%%%%%%%%%%%%%

\section{Experience}
\runsubsection{Backend software developer}
\descript{| Iquall S.A.}
\location{ Feb 2021 - Today}
\vspace{\topsep}
\begin{tightemize}
\item Currently working with Python and GraphQL to develop automated tests and an API to automate network operations. 
\end{tightemize}
\sectionsep
\runsubsection{Teaching Assistant}
\descript{| "Tecnologia de la informacion en organizaciones" - T. U. D. A. I}
\location{ Aug 2020 – Dic 2020}
\vspace{\topsep}
\begin{tightemize}
\item Course topics included Git and other collaborative dev tools.
\end{tightemize}
\sectionsep

%%%%%%%%%%%%%%%%%%%%%%%%%%%%%%%%%%%%%%
%		PROJECTS
%%%%%%%%%%%%%%%%%%%%%%%%%%%%%%%%%%%%%%
% Instead of saying in "currently being developed in R" I would say, "Used R to create location based data models showcasing behavioral patterns... "
\section{Projects}

\runsubsection{Graduate Thesis}
\descript{}
\begin{tightemize}
\item Currently working on visiting patterns and
automatic labeling of places from smartphone data. Using R to create location based data models to predict labels, and Shiny to visualize results.
\end{tightemize}
\sectionsep

\runsubsection{
\href{https://github.com/JCatrielLopez/InvOp-Comedor-UNICEN}{Queue-Based Monte Carlo Analysis to Improve Lunch Times on Campus}}
\descript{| College Coursework}
\begin{tightemize}
\item Final work for the "Operations Research" course.
\item We developed a Monte Carlo simulation of an everyday scenario for the dining hall on campus and proposed two solutions for the long queue times.
\item We analyzed how our solutions impacted the queue times and presented the conclusions on a Shiny dashboard.
\item Controlled through a R-Shiny dashboard, the simulation was implemented on Python utilizing the Simpy lib. End results were visualized by ggplot2 on R.
\end{tightemize}
\sectionsep

\runsubsection{\href{https://github.com/JCatrielLopez/magritte-cloud}{RAYMAN}}
\descript{| College Coursework}
\begin{tightemize}
\item Final work for the "Software engineering" course
\item A Feature Driven Design of a Personal Health Record app for Android.
\item I was involved in the design and implementation of the cloud server, its API and database.
\item Developed in Java and PostgreSQL, using Spring.
\end{tightemize}
\sectionsep


\sectionsep

%%%%%%%%%%%%%%%%%%%%%%%%%%%%%%%%%%%%%%
%     COURSES AND CERTIFICATIONS
%%%%%%%%%%%%%%%%%%%%%%%%%%%%%%%%%%%%%%

\section{Courses and Certifications} 

\begin{tabular}{rll}
2019   & \custombold{Fundamentals of Deep Learning for Computer Vision}\\
       & (NVIDIA Deep Learning Institute)\\
2018   & \custombold{Machine Learning para Seguridad en Redes y Detección}\\
       & \custombold{de Malware}\\
       & (Congreso Argentino de Ciencias de la Computación)\\
2017   & \custombold{Experto Universitario en Seguridad de la Información}\\
       & (Universidad Técnica Nacional) — \\
2017   & \custombold{Modelado de Sistemas Orientados a Objetos con Rational}\\
       & (U. N. I. C. E. N)\\
\end{tabular}
\sectionsep

\end{minipage} 
\end{document}

\documentclass[]{article}


